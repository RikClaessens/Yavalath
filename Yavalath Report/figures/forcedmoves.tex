% !TEX root = ../yavalathreport.tex
\begin{figure}[ht]
	\centering{
		\captionsetup[subfloat]{captionskip=0em}
		\subfloat[]{
			\scalebox{\yavalathboardsize}{
				\begin{tikzpicture}
					\yavalathboard
					\blackstone{4}{3};
					\blackstone{5}{5};
					\whitestone{6}{4};
					\whitestone{3}{4};
				\end{tikzpicture}
			}
		}
		\captionsetup[subfloat]{captionskip=0em}
		\subfloat[]{
			\scalebox{\yavalathboardsize}{
				\begin{tikzpicture}
					\yavalathboard
					\blackstone{4}{3};
					\blackstone{5}{5};
					\whitestone{6}{4};
					\whitestone{3}{4};
					\whitestonelabel{4}{4}{1};
					\blackstonelabelforced{5}{4}{2};
				\end{tikzpicture}
			}
		}
	}
	\caption{An opportunity for White to force a move for Black that completes a black line of three}
	\label{fig:forcedmoves}
\end{figure}